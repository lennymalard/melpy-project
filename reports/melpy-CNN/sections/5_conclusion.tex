{\color{gray}\hrule}
\begin{center}
\section{Conclusion}
\bigskip
\end{center}
{\color{gray}\hrule}
\vspace{0.5cm}

En conclusion, ce projet peut être considéré comme un succès, ayant atteint son objectif initial : 
implémenter, à partir de zéro et en s’appuyant sur NumPy, des outils permettant la création et 
l’entraînement de réseaux de neurones convolutifs. Cette avancée a également rempli son rôle 
pédagogique en me conduisant à étudier en profondeur les mécanismes fondamentaux des CNNs.

Cependant, nous avons observé une légère différence de performances 
entre notre implémentation et Keras, différence qui nécessitera une recherche plus approfondie pour être corrigée. 
Mais le principal axe d’amélioration réside dans l’optimisation des calculs. Cela pourrait être réalisé 
en intégrant d’autres bibliothèques telles que Numba, JAX ou Cython, permettant de compiler les modèles lors 
de l’entraînement et ainsi améliorer l’efficacité des calculs. \\