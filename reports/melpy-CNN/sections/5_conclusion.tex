{\color{gray}\hrule}
\begin{center}
\section{Conclusions}
\bigskip
\end{center}
{\color{gray}\hrule}
\vspace{0.5cm}

En conclusion, nous pouvons affirmer que ce projet a été un succès, ayant comme objectif initial d’implémenter 
la possibilité d’entraîner des CNNs avec Melpy. 

Nous avons observé une légère différence de performances 
entre notre implémentation et Keras, différence qui nécessitera une recherche plus approfondie pour être corrigée. 
Cependant, le principal axe d’amélioration réside dans l’optimisation des calculs. Cela pourrait être réalisé 
en intégrant d’autres bibliothèques telles que Numba, JAX ou Cython, permettant de compiler les modèles lors 
de l’entraînement et ainsi améliorer l’efficacité des calculs. \\