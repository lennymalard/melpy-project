{\color{gray}\hrule}
\begin{center}
\section{Conclusion}
\bigskip
\end{center}
{\color{gray}\hrule}
\vspace{0.5cm}

En conclusion, ce projet peut être considéré comme un succès, ayant atteint son objectif initial : 
implémenter, à partir de zéro et en s’appuyant de NumPy, des outils permettant la création et 
l’entraînement de réseaux de neurones convolutifs. Cette réalisation a également rempli son rôle 
pédagogique en m’amenant à apprendre en profondeur les mécanismes fondamentaux des CNNs. \\

Cependant, des différences ont été observées dans la vitesse de convergence des architectures, 
entre Melpy et Keras. Ces écarts ayant leurs origines connues, un travail plus poussé 
sera nécéssaire afin de les réduire. Toutefois, le principal axe d’amélioration reste l’optimisation des calculs. 
À cet effet, l’intégration de bibliothèques telles que Numba, JAX, ou Cython pourrait permettre de compiler les 
modèles lors de l’entraînement, améliorant ainsi la rapidité des calculs. 