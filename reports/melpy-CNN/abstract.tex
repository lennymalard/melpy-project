L’implémentation de réseaux de neurones artificiels à partir de zéro représente un véritable 
défi pédagogique. Cela exige une compréhension approfondie des concepts sous-jacents afin 
d’effectuer des inférences efficaces, qui pourront être utilisées dans des cas concrets. 
La création de Melpy, m’a permis d’apprendre à concevoir 
un framework en Python pouvant créer ces inférences, notamment pour des données tabulaires.
C'est donc dans cette optique d'apprentissage, que j'ai trouvé pertinent d’étendre les fonctionnalités de Melpy 
pour intégrer la vision par ordinateur, en utilisant des réseaux de neurones convolutifs (CNNs). 
En effet, ce type d’architecture requiert le traitement d'images, bien plus complexes a exploiter dans de 
simples perceptrons multicouches. \\

Étant donné que Melpy est basé sur l'utilisation de NumPy, de nouvelles couches spécifiques à ces 
architectures ont été ajoutées et optimisées en utilisant cette bibliothèque. Les résultats obtenus 
montrent que Melpy est désormais capable de produire des performances comparables à celles de Keras 
pour la classification d’images. En effet, le coût des architectures durant leur entrainenement, montrent
un écart notable d'une librairie à l'autre. Toutefois, cela a mit en évidence les limites en termes d’optimisation 
computationnelle de NumPy, dans sa forme purement Pythonesque, ce qui pourrait constituer un axe d’amélioration 
pour l’avenir.