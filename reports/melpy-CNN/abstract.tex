L’implémentation de réseaux de neurones artificiels à partir de zéro constitue un véritable défi 
pédagogique dans l’apprentissage du Deep Learning. Ce processus exige une compréhension approfondie 
des concepts fondamentaux, nécessaire pour transcrire ces connaissances dans la conception d’algorithmes 
optimisés. C’est dans cette optique d'apprentissage que j’ai créé Melpy, une bibliothèque 
de Deep Learning qui, à l’origine, n’implémentait que de simples perceptrons multicouches. Cependant, 
ce type d’architectures est limité aux données tabulaires simples. C'est donc pourquoi, j’ai entrepris d’implémenter 
les réseaux de neurones convolutifs afin d’étendre les possibilités offertes par Melpy et d’enrichir 
mes connaissances. \\


Étant donné que Melpy est basé sur l'utilisation de NumPy, de nouvelles couches spécifiques à ces 
architectures ont été ajoutées et optimisées en utilisant cette bibliothèque. Les résultats obtenus 
montrent que Melpy est désormais capable de produire des performances comparables à celles de Keras 
pour la classification d’images. En effet, le coût des architectures durant leur entrainenement, montrent
un écart notable d'une librairie à l'autre. Toutefois, cela a mit en évidence les limites en termes d’optimisation 
computationnelle de NumPy, dans sa forme purement Pythonesque, ce qui pourrait constituer un axe d’amélioration 
pour l’avenir.