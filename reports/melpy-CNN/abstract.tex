L’implémentation de réseaux de neurones artificiels à partir de zéro constitue un véritable défi 
pédagogique dans l’apprentissage du deep learning. Ce processus exige une compréhension approfondie 
des concepts fondamentaux, nécessaire pour transcrire ces connaissances dans la conception d’algorithmes 
optimisés. C’est dans cette optique d'apprentissage que j’ai créé Melpy, une bibliothèque 
de deep learning qui, à l’origine, n’implémentait que de simples perceptrons multicouches\cite{MLP} (MLPs). C'est donc pourquoi, 
j’ai entrepris d’implémenter les réseaux de neurones convolutif (CNNs) afin d’étendre les possibilités offertes par 
Melpy et d’enrichir mes connaissances. \\


Étant donné que Melpy repose sur l’utilisation de NumPy, de nouvelles couches spécifiques à ces architectures 
ont été ajoutées et optimisées en exploitant les fonctionnalités de cette bibliothèque. Les résultats obtenus 
montrent que Melpy est désormais capable de produire des modèles aussi précis que ceux réalisés avec Keras, 
dans la classification de jeux de données telles que MNIST\cite{MNIST} et CIFAR-10\cite{CIFAR10}. Cependant, ces résultats 
mettent également en évidence les limites de la bibliothèque en termes d’optimisation computationnelle, notamment dû à l'interprétation
du code en Python, et en termes de caclul différentiel. Ces deux éléments constituent ainsi les deux axes d'amélioration
principaux pour l'avenir.